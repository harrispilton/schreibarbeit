% "Messungen.tex" --- Kapitel 4 f�r das Template "Bachelor.tex"
%
% Wolfgang Sch�pf, November 2011
% letzte �nderung: Februar 2012


\chapter{Messungen}
\label{chap:messungen}
%
\section{Die mathematischen Umgebungen}
\label{sec:mathe}
%
hier fuege ich random shits ein!!!! (??)
Mathematische Formeln werden durch einen die Formel beschreibenden Text erzeugt. Hierzu dienen
die \emph{mathematischen Umgebungen}. Alles was innerhalb dieser Umgebungen steht, wird von
\LaTeX\ als Formel interpretiert. Der Vorteil von \LaTeX\ besteht darin, dass auch sehr komplizierte
Formeln und Gleichungen recht einfach und �bersichtlich geschrieben werden k�nnen.

Formeln k�nnen innerhalb von Textzeilen auftreten, wie $y = f(x) = x^2$, oder als abgesetzte 
Formeln, wie
%
\begin{equation}
\label{eq:abg}
\lim_{x \to 0} \frac{\ln \sin \pi x}{\ln \sin x} = 
     \lim_{x \to 0} \frac{\pi \frac{\cos \pi x}{\sin \pi x} }{ \frac{\cos x}{\sin x} } =
     \lim_{x \to 0} \frac{\pi \tan x}{\tan \pi x} = \quad \ldots \quad = 1 \, .
\end{equation}
%
Im ersten Fall wird die Formel zwischen zwei \verb+$+--Zeichen gesetzt, hier also

\verb+  $y = f(x) = x^2$+. 

F�r abgesetzte Formeln gibt es die \verb+equation+--Um\-gebung f�r einzeilige Formeln und die  
\verb+eqnarray+-- oder die \verb+align+--Umgebung f�r mehrzeilige Formeln.%
%
\footnote{Die \verb+align+--Umgebung wird anstelle der vielleicht bekannteren \verb+eqnarray+--Umgebung 
%
empfohlen, da letztere einige Nachteile hat \citetext{siehe z.\,B.~\citealp{Madsen2006}}.} 
Abgesetzte Formeln werden automatisch nummeriert und sollten immer mit einem Label zur 
Referenzierung versehen werden, wie dies in Gl.~(\ref{eq:abg}) und auch allen anderen geschehen ist.

\section{Beispiele f�r die Gestaltung von Formeln}
\label{sec:formeln}
%
Es gibt eine Vielzahl von mathematischen Befehlen und Gestaltungsm�glichkeiten, auf die hier nicht
alle eingegangen werden kann. Stattdessen werden einige Beispiele gezeigt, anhand deren Quelltext
sich die Bedeutung einiger M�glichkeiten erschlie�en l�sst. Gleichung~(\ref{eq:abg}) zeigt z.\,B.\
die Verwendung von ineinander geschachtelten Br�chen und verschiedener Funktionen, w�hrend in
Gl.~(\ref{eq:gauss}) eine Wurzel und auch Text (s. Abschnitt~\ref{subsec:variable}) innerhalb der 
Formel vorkommt:
%
\begin{equation}
\label{eq:gauss}
W_\text{G}(x) = \frac{1}{\sqrt{2 \pi \sigma^2}} \cdot \text{e}^{- \frac{(x - \bar{x})^2}{2 \sigma^2}}
\qquad \text{mit} \quad \bar{x} = N p , \, \sigma^2 = N p
\end{equation}
%
Die Gleichungen~(\ref{eq:k11}) und (\ref{eq:gamma}) zeigen eine mehrzeilige Formel und Gl.~(\ref{eq:matrix}) 
die Verwendung von Matrizen.%
%
\footnote{Gleichungen~(\ref{eq:k11}) und (\ref{eq:gamma}) wurden aus \citet{Khazimullin2011} entnommen.} 
%
In beiden F�llen wird eine Zeile jeweils durch \verb+\\+ beendet. Weitere mehr oder weniger komplizierte 
Formeln werden in den Gln.~(\ref{waer:nichtstat})--(\ref{eq:sin}) vorgestellt. Wie in den Beispielen zu 
sehen ist, k�nnen die verschiedenen mathematischen Ausdr�cke wie Br�che, Wurzeln, Summen, Integrale usw.\ 
beliebig ineinander verschachtelt werden.

\begin{align}
\label{eq:k11}
  k_{11} & = \frac{\epsilon_0 \epsilon_\text{a}}{\pi^2} U_\text{F}^2
    = \frac{\epsilon_0 \epsilon_\text{a}}{\pi^2}
      \frac{U_\text{on}^2}{\tau_\text{off}/\tau_\text{on}+1} \, , \\
%
\label{eq:gamma}
  \gamma_1 & = \frac{\pi^2}{d^2} k_{11} \tau_\text{off}
     = \frac{\epsilon_0 \epsilon_\text{a}}{d^2}
       \frac{\tau_\text{off} \, \tau_\text{on}}{\tau_\text{off} + \tau_\text{on}}
      U_\text{on}^2 \, .
\end{align}

\begin{equation}
\label{eq:matrix}
\mathcal{A} = \left( \begin{array}{ccc} 6 & 0 & 0 \\ 0 & 5 & 0 \\ 0 & 1 & 2 \end{array} \right)\, , \quad
\mathcal{B} = \left( \begin{array}{ccc} 1 & 2 & 3 \\ 2 & 1 & 0 \\ 4 & 3 & 0 \end{array} \right)\, \Rightarrow
\mathcal{A} \cdot \mathcal{B} = \left( \begin{array}{ccc} 6 & 12 & 18 \\ 10 & 5 & 0 \\ 10 & 7 & 0 \end{array} \right)
\end{equation}

\begin{eqnarray} 
\label{waer:nichtstat}
& & T(x,t) = T_\text{b} + (T_0 - T_\text{b}) \cdot \text{erf}(z) \\
\nonumber
&\text{mit} & \text{erf}(z) = \frac{2}{\sqrt{\pi}} \int_0^z \text{e}^{-y^2} \, \text{d}y
         \quad \text{und} \quad z = \frac{x}{2 \sqrt{m t}} 
\end{eqnarray}

\begin{equation}
\label{eq:n_e}
n_\text{e} = \frac{n_{\perp} n_{\|}}{\sqrt{n_{\perp}^2 \cos^2\theta(z) + n_{\|}^2 \sin^2 \theta(z)}} 
\, , \quad n_\text{o} = n_{\perp} \, .
\end{equation}

\begin{align}
\label{eq:bern}
W_\text{B}(x) & = \frac{N!}{x! (N - x)!} \cdot p^x (1-p)^{N - x} \\
\label{eq:bernmittel}
\bar{x}       & = \sum_{x = 0}^N x \, W_\text{B}(x) = Np \\
\label{eq:bernvarianz}
\sigma^2      & = \sum_{x = 0}^N (x - \bar{x})^2 \, W_\text{B}(x) = Np(1-p)
\end{align}

\begin{equation}
\label{eq:sin}
\sin \gamma \approx \gamma, \quad \cos \gamma \approx 1 - \frac{\gamma^2}{2}, \quad
     \sqrt{1 - (\frac{\gamma}{n})^2} \approx 1 - \frac{1}{2} \left( \frac{\gamma}{n} \right)^2
\end{equation}

Die mathematischen Umgebungen dienen nicht nur zum Setzen von Formeln, sondern auch zur Darstellung
einer Vielzahl von Sonderzeichen und griechischen Buchstaben. Beispiele sind $\alpha$, $\beta$ und 
$\Sigma$, aber auch $\nabla$, $\Downarrow$ und $\otimes$. Im Prinzip l�sst sich so ziemlich jedes
Symbol darstellen, man muss es nur finden. 

\section{Schriftarten in Formeln}
\label{sec:schrift}
%
\subsection{Variablen und Text, Indizes und Funktionen}
\label{subsec:variable}
%
Innerhalb einer mathematischen Umgebung werden Buchstaben und damit auch die daraus entstehenden
W�rter in $Italic$ gesetzt, was meistens auch gew�nscht ist. Es ist deshalb darauf zu achten, 
solche Variablen auch im fortlaufenden Text richtig zu setzen, also z.\,B.\ ``die Dicke $d$'' und 
\textbf{nicht} ``die Dicke d''. 

Will man umgekehrt Text innerhalb von Formeln schreiben, so muss man dies durch den Befehl 
\verb+\text{... text ...}+ erzwingen, wie in den Gln.~(\ref{eq:gauss}) und (\ref{waer:nichtstat}) 
geschehen. Ebenfalls nicht in $Italic$ gesetzt werden Funktionsnamen wie $\sin x$, $\arctan (2 \pi f)$ 
oder $\lim_{t \to \infty} f(t)$. Die meisten dieser Namen sind in \LaTeX\ definiert, so dass z.\,B.\ 
\verb+\sin+ verwendet werden kann. Das gleiche gilt f�r Indizes, welche Abk�rzungen darstellen, also 
z.\,B.\ ``$\tau_\text{off}$'' und \textbf{nicht} ``$\tau_{off}$'' (s.\ Gln.~\ref{eq:k11} und \ref{eq:gamma}).

\subsection{Gr��en, Zahlen und Einheiten}
\label{subsec:einheit}
%
Eine physikalische Gr��e besteht normalerweise aus einer Zahl und einer Einheit, welche zusammen geh�ren
und deshalb am Zeilenende nicht getrennt oder umgebrochen werden sollen. Auch ist der Abstand zwischen
beiden etwas kleiner als der normale Wortabstand, was durch das Einf�gen von \verb+\,+ erreicht wird. 

Es ist weiterhin darauf zu achten, dass im Gegensatz zu einer Variablen die Einheit nicht in $Italic$ 
gesetzt wird. Es muss also richtig hei�en ``die Dicke betr�gt $d = 15\,\si{\micro\metre}$ und nicht etwa 
``die Dicke betr�gt d = 15 $\mu$m'' und auch nicht ``die Dicke betr�gt $d = 15\,\mu m$. Dies wird erreicht 
durch \verb+$d = 15\,\si{\micro\metre}$+ oder auch durch \verb+$d = 15\,$\si{\micro\metre}+.%
%
\footnote{F�r den Befehl \verb+\si{<Einheit>}+ wird das Paket \verb+siunitx+ ben�tigt.} 
%
Ein weiteres Beispiel ist $g = 9.81\,\nicefrac{\text{m}}{\text{s}^2}$.
